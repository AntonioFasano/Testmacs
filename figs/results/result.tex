% This is the scratch of a template to be use by mix.R for random question generation
% Put many questions here so that mix.R can randomly extract only n of them 
% \CorrectChoice and \choice should always start on a new line
% mix.r wants \printanswers command, do not remove it.
% All questions need to have the same number of answers
% NOTE: The grade.R will replace the content of the first \fbox{...}, with:
% student details found in answer files and grades
% grades are obtained matching answer files with solution RData file.


\documentclass{exam}
\usepackage[utf8]{inputenc}

% This is an example of an external file where you possibly put your (maths) macros

\usepackage{amsmath}
\usepackage{amssymb}
\newcommand{\ex}{\mathbb{E}}
\newcommand{\one}{\mathbf{1}}
\newcommand{\zero}{\mathbf{0}}
\renewcommand{\ss}{\mathbf{S}}




\printanswers



\usepackage[
 reset,
 a4paper,
 landscape, twocolumn,
 left=1cm, right=1cm,
 top=2cm, bottom=1.5cm
 ]{geometry}

\pagestyle{headandfoot}
\runningheadrule
\firstpageheader{\textbf{Finance101}}{}{Page \thepage\ of \numpages}
\runningheader{\textbf{Finance101}}{}{Page \thepage\ of \numpages}

\firstpagefooter{}{}{}
\runningfooter{}{}{}

\begin{document}

% The content of the fbox below will be replaced in the result file with contact and grade data
\noindent
\fbox{\parbox{0.5\textwidth}
{John Doe jdoe$@$myschool.edu  nil Seat:\ pc14 Exam:\ 324 

February 15, 2019 %\today
\\\textbf{Your grade is}: $5$ (Good: 4, Wrong: 3)
}}


\bigskip\noindent 
Grading Policy: Good=$+2$, Wrong=$-1$, Missing=$0$

\begin{questions}

\question \textbf{(B)}
Given the  risk premium $\pi$ and the random payoff $X$, assume we approximate the utility $u$, with the two  Taylor expansions:
\begin{align*}
u(\ex X - \pi(X)) &\approx  u(\ex X) - u'(\ex X) \pi(X) \\
u(X) &\approx u(\ex X) + u'(\ex X) (X-\ex X) + \frac{1}{2} u''(\ex X)(X-\ex X)^2
\end{align*}
  
What is wrong with the approximations above? The calculation/application of:
\begin{choices}
\CorrectChoice  The utility function.
\choice  The expected value.
\choice  The expansion itself.
\end{choices}

\question \textbf{(NIL)}
Let  $\bar{r}$ be the vector of portfolio expected returns and  $\ss$ its covariance matrix. Set:
\begin{align*}
a&= \bar{r}'\ss^{-1} \bar{r}\\
b&= \bar{r}'\ss^{-1} \one \\
c&= \one'\ss^{-1} \one\\
d&= ac-b^2
\end{align*}

Which of the following is correct:
\begin{choices}
  \choice $
a( - b^2 +ac)\geq0
$
\CorrectChoice
$
a( - b^2 +ac) >0
$
\choice
$
a( - b^2 +ac) <0
$

\end{choices}

\question \textbf{(B)}
Prudential regulation:
\begin{choices}
\choice        Imposes maximum levels of risk and  minimal levels of capital buffer.
\CorrectChoice Imposes minimal levels of capital buffer, adequate to risks taken.
\choice        Imposes minimal levels of capital buffer, given the riskiness of credit exposures.
\end{choices}

\question \textbf{(A)}
The monetary base:
\begin{choices}
\choice Is formed  by the central bank's assets.
\CorrectChoice Consists of the currency in circulation and bank reserves.
\choice Is given by vault cash and central bank reserves.
\end{choices}

\question \textbf{(A)}
Shadow banking refers to:
\begin{choices}
\CorrectChoice Non-bank intermediaries providing services similar to traditional commercial banks.
\choice Non-bank intermediaries financing retail products manufactured by their parent companies.
\choice Non-bank intermediaries targeting borrowers not creditworthy for banks, at higher rates.
\end{choices}

\question \textbf{(B)}
In regard to Common Equity Tier 1 (CET 1), which of the following is wrong:
\begin{choices}
\choice CET 1 instruments have no maturity date, no right to be redeemed, and no specified coupon.
\CorrectChoice CET 1 instruments represents the less subordinated (most senior) claims.
\choice CET 1 instruments distributions (e.g. dividends) are wholly discretionary.
\end{choices}

\question \textbf{(NIL)}
The Liquidity Coverage ratio:
\begin{choices}
\choice Ensures that Tier 1 capital is sufficient to survive a serious one-month stress.
\CorrectChoice Is based on the ratio of high-quality liquid assets to projected net outflows.
\choice Is based on the ratio of high-quality liquid assets to projected net funding.
\end{choices}

\question \textbf{(A)}
In regard to Capital Conservation Buffer, which of the following is wrong:
\begin{choices}
\choice Institutions not meeting buffer requirements are limited in their distributions.
\CorrectChoice The buffer consists of CET 1 and Additional Tier 1 instruments.
\choice The buffer consists only of CET 1 instruments.
\end{choices}

\question \textbf{(C)}
Level 1 assets (used to calculate the Liquidity Coverage ratio) \emph{do not} include:
\begin{choices}
\choice Cash.
\choice Central bank reserves that can be drawn down in times of stress.
\CorrectChoice Corporate bonds.
\end{choices}
  
\end{questions}

\end{document}
